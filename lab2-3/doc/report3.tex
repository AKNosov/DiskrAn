%% * Copyright (c) toshunster, 2019
%% *
%% * All rights reserved.
%% *
%% * Redistribution and use in source and binary forms, with or without
%% * modification, are permitted provided that the following conditions are met:
%% *     * Redistributions of source code must retain the above copyright
%% *       notice, this list of conditions and the following disclaimer.
%% *     * Redistributions in binary form must reproduce the above copyright
%% *       notice, this list of conditions and the following disclaimer in the
%% *       documentation and/or other materials provided with the distribution.
%% *     * Neither the name of the w495 nor the
%% *       names of its contributors may be used to endorse or promote products
%% *       derived from this software without specific prior written permission.
%% *
%% * THIS SOFTWARE IS PROVIDED BY w495 ''AS IS'' AND ANY
%% * EXPRESS OR IMPLIED WARRANTIES, INCLUDING, BUT NOT LIMITED TO, THE IMPLIED
%% * WARRANTIES OF MERCHANTABILITY AND FITNESS FOR A PARTICULAR PURPOSE ARE
%% * DISCLAIMED. IN NO EVENT SHALL w495 BE LIABLE FOR ANY
%% * DIRECT, INDIRECT, INCIDENTAL, SPECIAL, EXEMPLARY, OR CONSEQUENTIAL DAMAGES
%% * (INCLUDING, BUT NOT LIMITED TO, PROCUREMENT OF SUBSTITUTE GOODS OR SERVICES;
%% * LOSS OF USE, DATA, OR PROFITS; OR BUSINESS INTERRUPTION) HOWEVER CAUSED AND
%% * ON ANY THEORY OF LIABILITY, WHETHER IN CONTRACT, STRICT LIABILITY, OR TORT
%% * (INCLUDING NEGLIGENCE OR OTHERWISE) ARISING IN ANY WAY OUT OF THE USE OF THIS
%% * SOFTWARE, EVEN IF ADVISED OF THE POSSIBILITY OF SUCH DAMAGE.

\documentclass[pdf, unicode, 12pt, a4paper,oneside,fleqn]{article}
\usepackage{log-style}
\begin{document}
\begin{titlepage}
    \begin{center}
    \bfseries
    
    {\Large Московский авиационный институт\\ (национальный исследовательский университет)
    
    }
    
    \vspace{48pt}
    
    {\large Факультет информационных технологий и прикладной математики
    }
    
    \vspace{36pt}
    
    {\large Кафедра вычислительной математики и~программирования
    
    }
    
    
    \vspace{48pt}
    
    Лабораторная работа \textnumero 3 по курсу \enquote{Дискретный анализ}
    
    \end{center}
    
    \vspace{72pt}
    
    \begin{flushright}
    \begin{tabular}{rl}
    Студент: & А.\,К. Носов \\
    Преподаватель: & А.\,А. Кухтичев \\
    Группа: & М8О-206Б-22 \\
    Дата: & \\
    Оценка: & \\
    Подпись: & \\
    \end{tabular}
    \end{flushright}
    
    \vfill
    
    \begin{center}
    \bfseries
    Москва, \the\year
    \end{center}
    \end{titlepage}
    
    \pagebreak

    \CWHeader{Лабораторная работа \textnumero 3}

    \CWProblem
    {Для реализации словаря из предыдущей лабораторной работы, необходимо провести исследование скорости выполнения и потребления оперативной памяти. В случае выявления ошибок или явных недочётов, требуется их исправить.
    Результатом лабораторной работы является отчёт, состоящий из:
    Дневника выполнения работы, в котором отражено что и когда делалось, какие средства использовались и какие результаты были достигнуты на каждом шаге выполнения лабораторной работы.
    Выводов о найденных недочётах.
    Сравнение работы исправленной программы с предыдущей версией.
    Общих выводов о выполнении лабораторной работы, полученном опыте.
    Минимальный набор используемых средств должен содержать утилиту gprof и библиотеку dmalloc, однако их можно заменять на любые другие аналогичные или более развитые утилиты (например, Valgrind или Shark) или добавлять к ним новые (например, gcov).}
    
    \CWHeader{Ход выполнения}
    Для выполнения лабораторной работы я использовал ноутбук операционной системой Ubuntu. Для выполнения анализа использовал программы Valgrind и gproof.
    \pagebreak

    \section{Valgrind}
    {Valgrind - инструментальное программное обеспечение, предназначенное для отладки использования памяти, обнаружения утечек памяти, а также профилирования. В ходе выполнения лабораторной работы утилита будет использована исключительно для отладки использования памяти. \\
    
    ==113422== Memcheck, a memory error detector \\
    ==113422== Copyright (C) 2002-2017, and GNU GPL'd, by Julian Seward et al. \\
    ==113422== Using Valgrind-3.18.1 and LibVEX; rerun with -h for copyright info \\
    ==113422== Command: ./lab2 \\
    ==113422==  \\
    + a 1234 \\
    OK \\
    + b 214324324 \\
    OK \\
    + c 23532523 \\
    OK \\
    - a \\
    OK \\
    - b \\
    OK \\
    c \\
    OK: 23532523 \\
    a \\
    NoSuchWord  \\
    ! Save asd \\
    OK \\
    - c \\
    OK  \\ 
    c \\
    NoSuchWord \\
    ! Load asd \\
    OK \\
    c \\ 
    OK: 23532523  \\
    ==113422== 
    ==113422== HEAP SUMMARY: \\
    ==113422==     in use at exit: 0 bytes in 0 blocks \\
    ==113422==   total heap usage: 13 allocs, 13 frees, 92,441 bytes allocated \\
    ==113422==  \\
    ==113422== All heap blocks were freed -- no leaks are possible \\ 
    ==113422==  \\
    ==113422== For lists of detected and suppressed errors, rerun with: -s \\
    ==113422== ERROR SUMMARY: 0 errors from 0 contexts (suppressed: 0 from 0) \\
    
    Как видим, Valgrind не обнаружил утечек памяти. Можно спать спокойно.}
    \pagebreak

\section{gprof}
gprof - это инструмент для профилирования программы. С помощью него мы можем отследить сколько времени и на каком участке кода выполнялась программа, тем самым выявляя слабые участки. Возьмем достаточно большой тест и применим утилиту gprof. \\

\begin{tabbing}
\%\quad\=cumulative\quad\=self\quad\=\qquad\qquad\qquad\=self\qquad\=total\\
time\quad\=seconds\quad\=seconds\quad\=calls\quad\=Ts/calls\quad\=Ts/calls\quad\=name\\
0.00\>0.00\>0.00\>265\>0.00\>0.00\>bool std::operator< <char, std::char\_traits<char>, std::allocator<char> >(std::\_\_cxx11::basic\_string<char, std::char\_traits<char>, std::allocator<char> > const\&, std::\_\_cxx11::basic\_string<char, std::char\_traits<char>, std::allocator<char> > const\&) \\

0.00\>0.00\>0.00\>174\>0.00\>0.00\>std::char\_traits<char>::compare(char const*, char const*, unsigned long) \\

0.00\>0.00\>0.00\>174\>0.00\>0.00\>\_\_gnu\_cxx::\_\_enable\_if<std::\_\_is\_char<char>::\_\_value, bool>::\_\_type std::operator==<char>(std::\_\_cxx11::basic\_string<char, std::char\_traits<char>, std::allocator<char> >\\

0.00\>0.00\>0.00\>174\>0.00\>0.00\>bool std::operator!=<char, std::char\_traits<char>, std::allocator<char> >(std::\_\_cxx11::basic\_string<char, std::char\_traits<char>, std::allocator<char> > const\&, std::\_\_cxx11::basic\_string<char, std::char\_traits<char>, std::allocator<char> > const\&)\\

0.00\>0.00\>0.00\>71\>0.00\>0.00\>bool std::operator><char, std::char\_traits<char>, std::allocator<char> >(std::\_\_cxx11::basic\_string<char, std::char\_traits<char>, std::allocator<char> > const\&, std::\_\_cxx11::basic\_string<char, std::char\_traits<char>, std::allocator<char> > const\&) \\

0.00\>0.00\>0.00\>54\>0.00\>0.00\>KV::~KV() \\

0.00\>0.00\>0.00\>40\>0.00\>0.00\>bool std::operator==<char, std::char\_traits<char>, std::allocator<char> >(std::\_\_cxx11::basic\_string<char, std::char\_traits<char>, std::allocator<char> > const\&, char const*) \\

0.00\>0.00\>0.00\>36\>0.00\>0.00\>KV::KV() \\

0.00\>0.00\>0.00\>29\>0.00\>0.00\>TAVLTree::Find(std::\_\_cxx11::basic\_string<char, std::char\_traits<char>, std::allocator<char> >) \\

0.00\>0.00\>0.00\>23\>0.00\>0.00\>KV::operator=(KV const\&) \\

0.00\>0.00\>0.00\>20\>0.00\>0.00\>TAVLTree::BalanceF(TNode*) \\

0.00\>0.00\>0.00\>20\>0.00\>0.00\>TAVLTree::Balance(TNode*) \\

0.00\>0.00\>0.00\>19\>0.00\>0.00\>TAVLTree::LeftRotate(TNode*) \\

0.00\>0.00\>0.00\>18\>0.00\>0.00\>TAVLTree::RightRotate(TNode*) \\

0.00\>0.00\>0.00\>18\>0.00\>0.00\>AVLTree::Insert(KV) \\

0.00\>0.00\>0.00\>18\>0.00\>0.00\>KV::KV(KV const\&) \\

0.00\>0.00\>0.00\>18\>0.00\>0.00\>TNode::TNode() \\

0.00\>0.00\>0.00\>18\>0.00\>0.00\>TNode::~TNode() \\

0.00\>0.00\>0.00\>11\>0.00\>0.00\>TAVLTree::Remove(TNode*) \\

0.00\>0.00\>0.00\>2\>0.00\>0.00\>\_\_static\_initialization\_and\_destruction\_0(int, int) \\

0.00\>0.00\>0.00\>1\>0.00\>0.00\>TAVLTree::Clear(TNode*) \\

0.00\>0.00\>0.00\>1\>0.00\>0.00\>TAVLTree::TAVLTree() \\

0.00\>0.00\>0.00\>1\>0.00\>0.00\>TAVLTree::~TAVLTree() \\
\\
Итак: \\
**Наиболее часто вызываемые функции:** \\
\\
* `\_\_gnu\_cxx::\_\_enable\_if<std::\_\_is\_char<char>::\_\_value, bool>::\_\_type\ std::operator==` \\ для сравнения строк на равенство \\
* `std::operator!=` для сравнения строк на неравенство \\
* `std::operator>` для сравнения строк \\
* `std::operator<` для сравнения строк типа `std::string` \\
\\
**Менее часто вызываемые функции:** \\
\\
* Деструктор класса `KV` (уничтожение объектов класса `KV`) \\ 
* Конструктор класса `KV` (создание объектов класса `KV`) \\ 
* `TAVLTree::Find` для поиска узла в двоичном дереве поиска \\
* `KV::operator=` для присваивания значений объектам класса `KV` \\
* `TAVLTree::BalanceF` для нахождения баланс-фактора \\
* Функции для перебалансировки дерева после вставки и удаления узлов \\ 
\\
**Функции, вызываемые редко:** \\ 
\\
* Конструктор и деструктор класса `TNode` (для узлов двоичного дерева поиска) \\
* Функции для статической инициализации и уничтожения в C++ \\ 
* Удаление всего дерева \\
\\

\end{tabbing}
\pagebreak

\section{Выводы}
Выполняя данную лабораторную работу, я познакомился с очень полезными инструментами и приобрел большой практический опыт. В дальнейшем я буду стараться использовать Valgrind и gprof как можно чаще.
\pagebreak

\begin{thebibliography}{99}

    \bibitem{Kormen}
    Томас\,Х.\,Кормен, Чарльз\,И.\,Лейзерсон, Рональд\,Л.\,Ривест, Клиффорд\,Штайн.
    {\itshape Алгоритмы: построение и анализ, 2-е издание.} --- Издательский дом \enquote{Вильямс}, 2007. Перевод с английского: И.\,В.\,Красиков, Н.\,А.\,Орехова, В.\,Н.\,Романов. --- 1296 с. (ISBN 5-8459-0857-4 (рус.))

    \bibitem{gost}
    \href{http://www.ifap.ru/library/gost/7052008.pdf}{Список использованных источников оформлять нужно по  ГОСТ Р 7.05-2008}
    
    \end{thebibliography}
    \pagebreak

\end{document}